\documentclass[12pt]{article}
\usepackage{calc}
\reversemarginpar
\usepackage{marvosym}
\usepackage[paper=letterpaper,      
            marginparwidth=1.2in,   
            marginparsep=.05in,      
            margin=1in,             
            includemp]{geometry}

\setlength{\parindent}{0in}
\usepackage{paralist}
\usepackage{fancyhdr,lastpage}
\pagestyle{fancy}
\fancyhf{}\renewcommand{\headrulewidth}{0pt}
\fancyfootoffset{\marginparsep+\marginparwidth}
\newlength{\footpageshift}
\setlength{\footpageshift}
          {0.5\textwidth+0.5\marginparsep+0.5\marginparwidth-2in}
\lfoot{\hspace{\footpageshift}%
       \parbox{4in}{\, \hfill %
                    \arabic{page} of \protect\pageref*{LastPage} 
                    \hfill \,}}

\usepackage{color,hyperref}
\definecolor{darkblue}{rgb}{0.0,0.0,0.3}
\hypersetup{colorlinks,breaklinks,
            linkcolor=darkblue,urlcolor=darkblue,
            anchorcolor=darkblue,citecolor=darkblue}

\newcommand{\makeheading}[1]%
        {\hspace*{-\marginparsep minus \marginparwidth}%
         \begin{minipage}[t]{\textwidth+\marginparwidth+\marginparsep}%
                {\large \bfseries #1}\\[-0.15\baselineskip]%
                 \rule{\columnwidth}{1pt}%
         \end{minipage}}

\renewcommand{\section}[2]%
        {\pagebreak[2]\vspace{1.3\baselineskip}%
         \phantomsection\addcontentsline{toc}{section}{#1}%
         \hspace{0in}%
         \marginpar{
         \raggedright \scshape #1}#2}

\newenvironment{outerlist}[1][\enskip\textbullet]%
        {\begin{itemize}[#1]}{\end{itemize}%
         \vspace{-.6\baselineskip}}

\newenvironment{lonelist}[1][\enskip\textbullet]%
        {\vspace{-\baselineskip}\begin{list}{#1}{%
        \setlength{\partopsep}{0pt}%
        \setlength{\topsep}{0pt}}}
        {\end{list}\vspace{-.6\baselineskip}}

\newenvironment{innerlist}[1][\enskip\textbullet]%
        {\begin{compactitem}[#1]}{\end{compactitem}}
\newcommand{\blankline}{\quad\pagebreak[2]}

\begin{document}
\makeheading{Kabitri Chattopadhyay}

\section{Contact Information}

\newlength{\rcollength}\setlength{\rcollength}{2.25in}%
%
\begin{tabular}[t]{@{}p{\textwidth-\rcollength}p{\rcollength}}
     Department of Physics        &  \Mobilefone{+49 162 885 7533} \\
     University of Oldenburg      &  \\
     Carl-von-Ossietzky-Str. 9-11 &  \\
     26129 Oldenburg, Germany     &  \href{kabitri.chattopadhyay@uni-oldenburg.de}
                                    {\nolinkurl{kabitri.chattopadhyay@uni-oldenburg.de }}  \\
                                 &  \href{kabitri.chattopadhyay@gmail.com}
                                    {\nolinkurl{kabitri.chattopadhyay@gmail.com}}  \\
\end{tabular}

\section{Citizenship}
%
Indian

\section{Date of birth}
%
08/05/1985

\section{Marital Status}
Married

\section{Education}

      Ph.D. in Physics, March 2013 - in progress  \\
      Department of Physics, University of Oldenburg, Oldenburg, Germany \\     
      Dissertation Topic : Optimization of spatial balancing and storage needs 
                              \hspace*{3.6cm} for large-scale power system 
                             integration of fluctuating \\ 
                             \hspace*{3.6cm}  solar energy : An investigation for European countries \\

      Advisor :  Prof. Dr. J{\"u}rgen Parisi \\
      Co-advisor: Dr. Detlev Heinemann and Dr. Elke Lorenz \\

\blankline

     Research Intern \\
     Max Planck Instit{\"u}t f{\"u}r Chemie, Mainz, Germany \\
     Topic: An investigation on dust transport events from the Saharan desert \\
     \hspace*{1.2cm} to the Mediterranean region using WRF-Chem and EMAC models \\
     January 2012 - February 2013

\blankline

      Junior Research Fellow \\
      Physical Research Laboratory, Ahmedabad, India. \\
      Topic: Tracer distribution study using chemical transport model MOZART \\ 
      July 2009 - November 2011.

\pagebreak

      M.Sc. in Atmospheric Sciences \\
      Calcutta University, India\\
      Thesis title: Prediction of pre-monsoon thunderstorm using Ant Colony \\
                         \hspace*{2.2cm} Optimization technique. \\
      August 2007 - May 2009. 

\blankline

      B.Sc. in Chemistry (Honours), Physics, Mathematics,  \\
      Scottish Church College, Calcutta University, India \\
      August 2004 - July 2007.

\section{Awards} 
%
  First class first in M.Sc Atmospheric Sciences, 2009.

\blankline

   PRL Doctoral Fellowship, 2009 \\
   Physical Research Laboratory, Ahmedabad, India.
   
\blankline

   TOEFL score: 101 \\
   Date of examination: 12.08.2011.
   
\section{Languages}
  Bengali (excellent proficiency), English (excellent proficiency), German (basic).

\section{Publications}
 
\begin{enumerate}

  \item {\sl K. Chattopadhyay}\footnote{Last name changed to Chattopadhyay from Nag in 2014}, A. Kies, 
            E. Lorenz, L. von Bremen, and D. Heinemann :
        {\sf The impact of different PV module configurations on storage and additional balancing needs 
              for a fully renewable European power system,}               \\
              Renewable Energy {\bf(under review)}.
              
  \item A. Kies, {\sl K. Nag}, L. von Bremen, E. Lorenz, and D. Heinemann :
        {\sf Investigation of balancing effects in long term renewable energy feed-in with respect to the transmission grid,} \\
              Adv. Sci. Res., {\bf 12}, 91-95, 2015.
  
\end{enumerate}

\section{Conference Presentations}

\begin{enumerate}

\item A. Kies, L. von Bremen, B. Schyska, {\sl K. Chattopadhyay}, E. Lorenz, and  D. Heinemann :
        {\sf The Cost-Optimal Distribution of Wind and Solar Generation Facilities in a Simplified Highly 
               Renewable European Power System}, \\
            EGU2016 (Oral),
            Vienna, Austria (19 April 2016).
            
\item  A. Kies, {\sl K. Nag}, L. von Bremen, E. Lorenz, and D. Heinemann :
           {\sf Effects of Scandinavian hydro power on storage needs in a fully renewable 
                 European power system for various transmission capacity scenarios}, \\
            EGU2015 (Oral), 
            Vienna, Austria (14 April 2015).
            
\item A. Kies, L. von Bremen, D. Heinemann, {\sl K. Nag},  and  E. Lorenz : 
           {\sf Impacts of North African CSP and Norwegian hydro power on storage needs in a European 
                 energy system dominated by Renewables.}, \\
            13th Wind Integration Workshop (Oral),
            Berlin, Germany (12 November 2014).
                                                                                                  
            
\item {\sl K. Nag}, A. Kies, E. Lorenz, L. von Bremen, and D.Heinemann :
           {\sf Simulation of long term solar power feed-in and solar balancing potential in European countries}, \\
           14th EMS/10th ECAC conference (Oral),
           Prague, Czech Republic (07 October 2014).
 
\item A. Kies, {\sl K. Nag}, L. von Bremen, E. Lorenz, and D. Heinemann :
           {\sf Simulation of long term renewable energy feed-in for European power system studies}, \\
           EGU2014 (Poster),
           Vienna, Austria (28 April 2014).
 
\end{enumerate}

\section{Work experience}

\begin{itemize}

  \item Active involvement in RESTORE 2050: A project that investigated the balancing needs, the importance of the
           transmission grid expansion, and the methods of load management for a sustainable European power system 
           in 2050. The project combines meta analyses and in-depth system studies using model simulations. 
  \item Expertise in deriving solar irradiance (and by extension PV and CSP power) from clear sky model and combining 
           Meteosat First- and Second Generation satellite images to retrieve cloud information.
  \item Proficiency in modeling the integration of variable renewable sources to the power system and simulating the
           storage and other balancing needs. 
  \item Special competence in determining the influence of different PV module configurations (angle of inclinations and
           azimuth angles of orientation) on storage and balancing needs.
  \item Long term analysis of wind and solar power data on different temporal scales to identify their nature of variabilities 
          and to further investigate how the mix of these two renewable sources can influence the balancing needs.

\end{itemize}

\newpage

\section{Computer Skills}
\begin{enumerate}

  \item Proficiency in scientific programming with C  and MATLAB \\
           Familiar with Fortran 77/90 \\
           Familiar with high performance computing. \\
           Familiar with bash shell scripts. \\        

  \item Software Tools : IDL, Pstricks

  \item Applications:  \LaTeX, Open-Office

  \item Operating Systems:  Unix/Linux, OSX.\\
  
  \item Familiar with special binary files like NetCDF, HDF etc. \\

\end{enumerate}

\section{References}

\begin{itemize}
  \item Dr. Detlev Heinemann \\
           University of Oldenburg \\
           Institute of Physics, Energy Meteorology Group \\
           Carl-von Ossietzky Str. 9-11, 26129 Oldenburg, Germany \\
           Email: detlev.heinemann@uni-oldenburg.de \\
  \item Dr. Elke Lorenz \\
           Group Manager Energy Meteorology \\
           Dept. Quality Assurance of PV Modules and Power Plants \\
           Fraunhofer Institute for Solar Energy Systems ISE \\
           Heidenhofstrasse 2, 79110 Freiburg, Germany \\
           Email: elke.lorenz@ise.fraunhofer.de \\
\end{itemize}

\end{document}